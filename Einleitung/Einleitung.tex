\graphicspath{{Einleitung/img/}}
\chapter{Einleitung}\label{sec:einleitung}

\section{Motivation}\label{sec:motivation}

Lorem ipsum dolor sit amet, consetetur sadipscing elitr, sed diam nonumy eirmod
tempor invidunt ut labore et dolore magna aliquyam erat, sed diam voluptua. At
vero eos et accusam et justo duo dolores et ea rebum. Stet clita kasd
gubergren, no sea takimata sanctus est Lorem ipsum dolor sit amet. Lorem ipsum
dolor sit amet, consetetur sadipscing elitr, sed diam nonumy eirmod tempor
invidunt ut labore et dolore magna aliquyam erat, sed diam voluptua. At vero
eos et accusam et justo duo dolores et ea rebum. Stet clita kasd gubergren, no
sea takimata sanctus est Lorem ipsum dolor sit amet.

% section motivation (end)

\section{Ziel der Arbeit}\label{sec:ziel_der_arbeit}
Was passiert wenn ich versuche sehr viel zu schreiben. Und dabei ein bisschen langsamer bin als gedacht. Und dann schreibe ich noch ein bisschen mehr um zu schauen was passiert. Wann fängt er an zu speichern. Wie lange dauert es bis er wieder neu builded.

\ldots

% section ziel_der_arbeit (end)

\section{Vorgehensweise}\label{sec:vorgehensweise}
Deru said Deep Learning is hard \autocite{deruDeepLearningMit2020}.
Allerdings meint Goodfellow, dass es nicht so schwer ist \autocite{goodfellowDeepLearning2016}
Weiterhin meint Roberts, dass es sehr komplex sein kann \autocite{robertsPrinciplesDeepLearning2022}.
Wohingegen Prince sagt: 
\begin{quoting}
Deep Learning ist nicht so komplex wie die meisten es annehmen. Nach einiger Zeit und ein bisschen Aufwand merkt man schnell, dass man sehr schnell erfolge feiern kann im Bereich der Neuronalen Netze \autocite{princeUnderstandingDeepLearning2023}.
\end{quoting}
Die Abbildung~\ref{fig:google3d} zeigt das Google 3D Logo.
\begin{figure}[ht]
  \centering
  \includegraphics[scale=0.25]{google3d}
  \caption{Google 3D Logo}\label{fig:google3d}
\end{figure}
% section vorgehensweise (end)
